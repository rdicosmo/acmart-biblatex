%% For double-blind review submission, w/o CCS and ACM Reference (max submission space)
% \documentclass[acmsmall,review,anonymous,natbib=false]{acmart}
% \settopmatter{printfolios=true,printccs=false,printacmref=false}
\documentclass[acmsmall,review,anonymous,natbib=false]{acmart}\settopmatter{printfolios=true,printccs=false}
%% For double-blind review submission, w/ CCS and ACM Reference
%\documentclass[acmsmall,review,anonymous]{acmart}\settopmatter{printfolios=true}
%% For single-blind review submission, w/o CCS and ACM Reference (max submission space)
%\documentclass[acmsmall,review]{acmart}\settopmatter{printfolios=true,printccs=false,printacmref=false}
%% For single-blind review submission, w/ CCS and ACM Reference
%\documentclass[acmsmall,review]{acmart}\settopmatter{printfolios=true}
%% For final camera-ready submission, w/ required CCS and ACM Reference
%\documentclass[acmsmall]{acmart}\settopmatter{}


%% Journal information
%% Supplied to authors by publisher for camera-ready submission;
%% use defaults for review submission.
\acmJournal{PACMPL}
\acmVolume{1}
\acmNumber{CONF} % CONF = POPL or ICFP or OOPSLA
\acmArticle{1}
\acmYear{2018}
\acmMonth{1}
\acmDOI{} % \acmDOI{10.1145/nnnnnnn.nnnnnnn}
\startPage{1}

%% Copyright information
%% Supplied to authors (based on authors' rights management selection;
%% see authors.acm.org) by publisher for camera-ready submission;
%% use 'none' for review submission.
\setcopyright{none}
%\setcopyright{acmcopyright}
%\setcopyright{acmlicensed}
%\setcopyright{rightsretained}
%\copyrightyear{2018}           %% If different from \acmYear

%% Bibliography style
\RequirePackage[
  style=acmauthoryear,
  backend=biber,
  natbib=false         % natbib must be disabled to retain style compatibility with acmart
]{biblatex}

\addbibresource{bib.bib}
\addbibresource{sample-base.bib}



%%%%%%%%%%%%%%%%%%%%%%%%%%%%%%%%%%%%%%%%%%%%%%%%%%%%%%%%%%%%%%%%%%%%%%
%% Note: Authors migrating a paper from PACMPL format to traditional
%% SIGPLAN proceedings format must update the '\documentclass' and
%% topmatter commands above; see 'acmart-sigplanproc-template.tex'.
%%%%%%%%%%%%%%%%%%%%%%%%%%%%%%%%%%%%%%%%%%%%%%%%%%%%%%%%%%%%%%%%%%%%%%


%% Some recommended packages.
\usepackage{booktabs}   %% For formal tables:
                        %% http://ctan.org/pkg/booktabs
\usepackage{subcaption} %% For complex figures with subfigures/subcaptions
                        %% http://ctan.org/pkg/subcaption

%
% Keep main directory clean: images are in ./images
%
\graphicspath{{./}{./images/}}

\begin{document}

%% Title information
\title[Short Title]{Full Title}         %% [Short Title] is optional;
                                        %% when present, will be used in
                                        %% header instead of Full Title.
\titlenote{with title note}             %% \titlenote is optional;
                                        %% can be repeated if necessary;
                                        %% contents suppressed with 'anonymous'
\subtitle{Subtitle}                     %% \subtitle is optional
\subtitlenote{with subtitle note}       %% \subtitlenote is optional;
                                        %% can be repeated if necessary;
                                        %% contents suppressed with 'anonymous'


%% Author information
%% Contents and number of authors suppressed with 'anonymous'.
%% Each author should be introduced by \author, followed by
%% \authornote (optional), \orcid (optional), \affiliation, and
%% \email.
%% An author may have multiple affiliations and/or emails; repeat the
%% appropriate command.
%% Many elements are not rendered, but should be provided for metadata
%% extraction tools.

%% Author with single affiliation.
\author{First1 Last1}
\authornote{with author1 note}          %% \authornote is optional;
                                        %% can be repeated if necessary
\orcid{nnnn-nnnn-nnnn-nnnn}             %% \orcid is optional
\affiliation{
  \position{Position1}
  \department{Department1}              %% \department is recommended
  \institution{Institution1}            %% \institution is required
  \streetaddress{Street1 Address1}
  \city{City1}
  \state{State1}
  \postcode{Post-Code1}
  \country{Country1}                    %% \country is recommended
}
\email{first1.last1@inst1.edu}          %% \email is recommended

%% Author with two affiliations and emails.
\author{First2 Last2}
\authornote{with author2 note}          %% \authornote is optional;
                                        %% can be repeated if necessary
\orcid{nnnn-nnnn-nnnn-nnnn}             %% \orcid is optional
\affiliation{
  \position{Position2a}
  \department{Department2a}             %% \department is recommended
  \institution{Institution2a}           %% \institution is required
  \streetaddress{Street2a Address2a}
  \city{City2a}
  \state{State2a}
  \postcode{Post-Code2a}
  \country{Country2a}                   %% \country is recommended
}
\email{first2.last2@inst2a.com}         %% \email is recommended
\affiliation{
  \position{Position2b}
  \department{Department2b}             %% \department is recommended
  \institution{Institution2b}           %% \institution is required
  \streetaddress{Street3b Address2b}
  \city{City2b}
  \state{State2b}
  \postcode{Post-Code2b}
  \country{Country2b}                   %% \country is recommended
}
\email{first2.last2@inst2b.org}         %% \email is recommended


%% Abstract
%% Note: \begin{abstract}...\end{abstract} environment must come
%% before \maketitle command
\begin{abstract}
Text of abstract \ldots.
\end{abstract}


%% 2012 ACM Computing Classification System (CSS) concepts
%% Generate at 'http://dl.acm.org/ccs/ccs.cfm'.
\begin{CCSXML}
<ccs2012>
<concept>
<concept_id>10011007.10011006.10011008</concept_id>
<concept_desc>Software and its engineering~General programming languages</concept_desc>
<concept_significance>500</concept_significance>
</concept>
<concept>
<concept_id>10003456.10003457.10003521.10003525</concept_id>
<concept_desc>Social and professional topics~History of programming languages</concept_desc>
<concept_significance>300</concept_significance>
</concept>
</ccs2012>
\end{CCSXML}

\ccsdesc[500]{Software and its engineering~General programming languages}
\ccsdesc[300]{Social and professional topics~History of programming languages}
%% End of generated code


%% Keywords
%% comma separated list
\keywords{keyword1, keyword2, keyword3}  %% \keywords are mandatory in final camera-ready submission


%% \maketitle
%% Note: \maketitle command must come after title commands, author
%% commands, abstract environment, Computing Classification System
%% environment and commands, and keywords command.
\maketitle


\section{Introduction}

Test brackets:~\cite{gf-tag-sound-repo,ad-wood-2003}\\
Test repeated items:~\cite{897367,Buss:1987:VTB:897367}\\
Test brackets*:~\cite*{gf-tag-sound-repo,ad-wood-2003}\\
Test repeated items*:~\cite*{897367,Buss:1987:VTB:897367}\\
Test citeauthor: ~\citeauthor{ad-wood-2003}\\
Test citeyear: ~\citeyear{ad-wood-2003}\\
Test citeyearpar: ~\citeyearpar{ad-wood-2003}\\
Test textcite: ~\textcite{ad-wood-2003}\\
Test textcite multiple: ~\textcite{ad-wood-2003,897367,Buss:1987:VTB:897367}\\

\subsubsection{More BibLaTeX Tests}

From \url{https://tex.stackexchange.com/a/27615/133551}.

Filler text \parencite{Knuth98}.
Filler text \parencite{Lamport:LaTeX}. Filler text \parencite{Amsthm15}. \\
Filler text \parencite[See][23]{Knuth98}.
Filler text \parencite[1--10]{Lamport:LaTeX}. \\
\textcite{knuth:texbook} and \textcite{Knuth97}.
\textcite{Knuth98} and \textcite{knuth:texbook}. \\
\textcite{Knuth98} and \textcite{Lamport:LaTeX} and \textcite{Amsthm15}. \\
\textcite[e.g.][]{Knuth98} and \textcite[10]{Lamport:LaTeX}. \\
Filler text.\footcite[23]{Knuth98} Filler text.\footcite[1--10]{Knuth97}
Filler text.\footnote{\smartcite[10--15]{Goossens:1999:LWC:553897}}

\textbf{Unqualified citation lists}

\textcite{knuth:texbook,Knuth97,Knuth98,knuth:texbook} showed that... \\
\textcite[e.g.][10--15]{Knuth98,Knuth97,Amsthm15} showed that...\\
Filler text \parencite[See][for example]{Knuth98,Knuth97,Amsthm15}. \\
Filler text \parencite[etc.]{knuth:texbook,Knuth97,Knuth98,knuth:texbook}.

\textbf{Qualified citation lists}

\textcites{Knuth98}{Knuth97} showed that...
\textcites(See)(){Knuth98}[cf.][]{Knuth97}. \\
\textcites(See)()[e.g.][15]{Knuth98}[cf.][10]{Knuth97} \\
\parencites(See)()[10--15]{Knuth98}[cf.][10]{Knuth97} \\
\parencites{knuth:texbook,Knuth97}[10--11]{Knuth98,knuth:texbook}

\textbf{Mix of qualified and unqualified citation lists}

\textcites(See)()[e.g.][]{Knuth98}[10]{JCohen96,Goossens:1999:LWC:553897} \\
\textcites[e.g.][]{Knuth98,Knuth97}[10]{Goossens:1999:LWC:553897} \\
\textcites[10]{Knuth98}{Knuth97}[cf.][]{JCohen96} \\
\textcites[15]{Knuth98}[cf.][10]{JCohen96,Goossens:1999:LWC:553897}

\input{samplebody}

%% Acknowledgments
\begin{acks}                            %% acks environment is optional
                                        %% contents suppressed with 'anonymous'
  %% Commands \grantsponsor{<sponsorID>}{<name>}{<url>} and
  %% \grantnum[<url>]{<sponsorID>}{<number>} should be used to
  %% acknowledge financial support and will be used by metadata
  %% extraction tools.
  This material is based upon work supported by the
  \grantsponsor{GS100000001}{National Science
    Foundation}{http://dx.doi.org/10.13039/100000001} under Grant
  No.~\grantnum{GS100000001}{nnnnnnn} and Grant
  No.~\grantnum{GS100000001}{mmmmmmm}.  Any opinions, findings, and
  conclusions or recommendations expressed in this material are those
  of the author and do not necessarily reflect the views of the
  National Science Foundation.
\end{acks}


%% Bibliography

\nocite{*}

\printbibliography

%% Appendix
\appendix
\section{Appendix}

Text of appendix \ldots

\end{document}
